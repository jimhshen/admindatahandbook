
\newcommand*{\thead}[1]{\bfseries\hspace*{\fill}{#1}\hspace*{\fill}}
\begin{center}
\begin{table}[H]
\caption{\label{table:dp_usecases} Considerations when deciding whether to use differential privacy for a particular use case.}
\fontsize{8}{9}\selectfont
% \begin{tabular}{ p{1.3in} p{1.3in} p{1.3in} }
\begin{tabularx}{1\textwidth}{YYY}
\toprule
Use cases where DP is more likely to be appropriate & Use cases where DP is not appropriate & Use cases where DP is challenging \\
\midrule
\begin{itemize}
\item Informational harm derives from making inferences about individuals or small groups
\item Intended use is statistical analysis of population or large groups
\item Sensitivity of information is high
\item Information and analyses are highly structured
\item Datasets are large
\item Types of analyses to be conducted are known in advance
\item Composition effects are important
\item Release of (low-dimensional) synthetic data is acceptable or preferred
\end{itemize}
&
\begin{itemize}
\item Informational harm derives from making inferences about large groups
\item Intended use is individual inference or individual intervention
\item Intended control is purpose limitation
\item Intended control is computation limitation\textsuperscript{1}
\item Datasets are very small (e.g., less than a few dozen observations)
\end{itemize}
&
\begin{itemize}
    \item Supporting data linking
    \item Supporting data cleaning
    \item Estimating complex statistical models efficiently
    \item Datasets are small (e.g., dozens to thousands of observations)\textsuperscript{2}
    \item Differentially private analysis not yet available
    \item Intended output is high-dimensional synthetic data
\end{itemize}
\\
\bottomrule
\end{tabularx}
\vspace{1pt}
\raggedright
{\fontsize{8}{9}\selectfont \textsuperscript{1}A control on computation is designed to ``limit the direct operations that can be meaningfully performed on data. Commonly used examples are file-level encryption and interactive analysis systems or model servers. Emerging approaches include secure multiparty computation, functional encryption, homomorphic encryption, and secure public ledgers, eg blockchain technologies.'' \citep{AltmanPracticalApproaches}.\\
\textsuperscript{2}For a real-world example, see the Opportunity Atlas case study presented in Section~\ref{subsec:opportunity_atlas}.}
\end{table}
\end{center}
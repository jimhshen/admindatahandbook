\documentclass[11pt]{book}
\usepackage[utf8]{inputenc}

%Spacing
\usepackage{setspace}
\setstretch{1.15}

% Font:
\renewcommand{\rmdefault}{bch}
\renewcommand{\sfdefault}{phv}
%Define font size between huge and Huge
\makeatletter 
\newcommand\semiHuge{\@setfontsize\semiHuge{22.72}{27.38}}
\makeatother

% page margins 
\usepackage[paperheight=9in,paperwidth=7in,top=1in,bottom=1in,right=1in,left=1in]{geometry}
\usepackage{changepage}
%\usepackage[cam,letter,center]{crop}

% lists:
%\usepackage{enumitem}
% tables
\usepackage{booktabs}
% use one of those if needed for referencing figures etc.
%\usepackage{varioref}
%\usepackage{hyperref}
%\usepackage{cleveref}

% headers and footers
\usepackage{fancyhdr}
% Section and chapter titles
\usepackage{titling}

%Headings and chapter titles
\usepackage{titlesec}
%Make "Chapter 1" all caps
\renewcommand{\chaptername}{CHAPTER}



%\setuphead[chapter][align={flushleft, nothyphenated, verytolerant}]
%Format chapter titles on chapter title pages
\titleformat{\chapter}[display]
  {\Large\sffamily}{\chaptertitlename\ \thechapter}{14pt}{\semiHuge\bfseries}
\titlespacing*{\chapter}{0pt}{0pt}{40pt}
%Set heading styles
\titleformat*{\section}{\Large\sffamily\bfseries\parindent0pt}
\titleformat*{\subsection}{\large\sffamily\bfseries}
\titleformat*{\subsubsection}{\normalsize\sffamily\bfseries}
%\renewcommand{\maketitlehooka}{\rmfamily}


%% Setting headers and footers 
% use after geometry to get right size
\pagestyle{fancy}
\fancyhf{}% Clear header/footer
%Headers:
\makeatletter
\fancyhead[RO]{\sffamily \scriptsize \scshape Using Administrative Data for Research and Evidence-Based Policy}
\fancyhead[LO]{ }
\fancyhead[LE]{\sffamily \scshape \scriptsize \if@mainmatter Chapter \thechapter\fi}
\makeatother
\renewcommand{\chaptermark}[1]{\markboth{#1}{}}


%Footers: 
\makeatletter
\renewcommand{\pagenumbering}[1]{\gdef\thepage{\csname
@#1\endcsname\c@page}}
\makeatother
\fancyfoot{} % clear all footer fields
\fancyfoot[LE,RO]{\small \thepage}
\fancyfoot[LO,CE]{}
\fancyfoot[CO,RE]{}
\renewcommand{\headrulewidth}{0.0pt}
%Make chapter title pages have the same page numbers as the rest:
\fancypagestyle{plain}{% % <-- this is new
  \fancyhf{} 
  \fancyfoot[LE,RO]{\small \thepage} % same placement as with page style "fancy"
  \renewcommand{\headrulewidth}{0pt}}

%% Title page:
% Formatting the title of the book to flush left
%\pretitle{ \begin{flushleft}\LARGE\sffamily  }
%\posttitle{ \par \end{flushleft} \vskip 2em}

% Formatting the author list

\preauthor{ }
\DeclareRobustCommand{\authorthing}
{\begin{flushleft}\Large
\begin{adjustwidth}{-.16cm}{}
    \begin{tabular}{l}
    %\setlength{\tabcolsep}{1pt}
    Shawn Cole \\ Iqbal Dhaliwal \\ Anja Sautmann \\ Lars Vilhuber 
\end{tabular}
\end{adjustwidth}
\end{flushleft}
}
\postauthor{}

%Title here
% Title page
\pretitle{ \begin{flushleft}\LARGE\sffamily  }

\title{ \Large \uppercase{Handbook on} \\\semiHuge\bfseries {Using Administrative Data \\ for Research and \\ Evidence-based Policy}}
\posttitle{ \par \end{flushleft} \vskip 2em} 

\author{\sffamily \Large \authorthing}

\date{ }



% Chapter titles with authors in ToC and chapter 
\makeatletter

\newcommand{\chapterauthor}[1]{\authortoc{#1}\printchapterauthor{#1}}

\newcommand{\printchapterauthor}[1]{%
    {\parindent0pt\vspace*{-35pt}%
    \linespread{2}\large\sffamily #1%
    \par\nobreak\vspace*{35pt}}
    \@afterheading%
}
\newcommand{\authortoc}[1]{%
    \addtocontents{toc}{\vskip-10pt}%
    \addtocontents{toc}{%
    \protect\contentsline{chapter}%
    {\parindent0pt\hskip1.5em \mdseries\protect\rmfamily #1}{}{}}
    \addtocontents{toc}{\vskip2pt}%
}
\makeatother

\setlength{\parindent}{0cm}
\setlength{\parskip}{.2cm plus 1mm minus 1mm}


% TOC
\usepackage{tocloft}
\renewcommand{\cfttoctitlefont}{\sffamily\huge\bfseries}
\renewcommand{\cftchapfont}{\rmfamily \bfseries}
\renewcommand{\cftchappagefont}{\normalfont \small}
%\renewcommand{\cftchapdotsep}{\cftdotsep} 
%\renewcommand{\cftpartleader}{\cftdotfill{}} % for parts


%this keeps the toc from adding extra stuff (subsections)
\setcounter{tocdepth}{0}
\renewcommand{\chaptermark}[1]{%
  \ifnum\value{chapter}>0
    \markboth{Chapter \thechapter{}: #1}{}%
  \else
    \markboth{#1}{}%
  \fi}




%%%%%%%%%%%%%%%%%%%%%%%%%%%%%%%%%%%%%%%%%%%%%%%%%%%%%%%%%%%%%%%%%%%%%%%%%%%%
% Document start
\begin{document}

\frontmatter
\pagenumbering{arabic}

 % Title page rendering
\maketitle

\newpage

% TOC
\tableofcontents 


\newpage
\pagestyle{fancy}

\mainmatter


% Testing chapters
% this will eventually be replace by files for each chapter

%\include{Introduction.tex}
%\include{Chapter1.tex}

\chapter[Increasing Access to Administrative Data: Why and How?]{Increasing Access to \\ Administrative Data: \\ Why and How?}

\chapterauthor{Shawn Cole, Iqbal Dhaliwal, Anja Sautmann, Lars Vilhuber}
\hrulefill

\section{Summary}

If there's an executive summary it would be here.
\newpage
\section{The Potential of Administrative Data for Research and Policymaking}

Over the course of our careers, we, the editors of this Handbook, have been witness to extraordinary changes in economics and economic research. One of them has been the rise of research in applied microeconomics and development economics that focuses on working closely with policymaking bodies and creating an evidence base for better social programming. Two key factors have contributed to this trend: increased availability of new data sources, and the rapid growth in the use of experiments (randomized control trials or randomized evaluations) in the social sciences. These developments have enabled many new avenues of research, such as how behavioral factors affect optimal policy design, how to credibly evaluate long-run effects of landmark social programs, or how to make changes to these programs informed by a better understanding of the levers of impact. In the process, they have dramatically improved the quality and breadth of evidence used to inform policy.

Yet it is also our experience that this type of research is in many cases conducted by carrying out complex, time-consuming, and costly original data collection: for example, the large-scale surveys that accompany many randomized control trials typically consume a large share of the financial and staff resources devoted to the research project overall. A lack of relevant, reliable, and comprehensive data that researchers can access has been a limiting factor for new studies and consequently the spread of evidence-informed policy.

At the same time, there are a wide variety of data sets already in existence, such as census records, banking data, employment information, or GPS records, which could dramatically reduce the cost and complexity of policy-relevant research – including randomized control trials – and speed up the formation of an evidence base for policymaking. Administrative data, sometimes referred to as organic data (Groves 2011) because they are generated as part of normal business processes, often contain comprehensive, objective data about large populations of interest. Decision-makers at firms and in government are already using such data to better understand problems and issues of populations of interest, and based on such evidence-based analytics, new policies are implemented or new questions are defined.

Beyond these uses, carefully designed systematic research with administrative data, often partnering academic researchers with firms and governments, may carry out analyses, conduct experiments, and develop and field supplemental surveys to test specific mechanisms or hypotheses. This type of innovative research could dramatically expand the types of insights gained from the data. An increasing fraction of published papers in economics uses administrative data (see Figure 1.1, Chetty (2012)). However, researcher access to administrative data sets remains difficult and idiosyncratic (Card et al. 2010). This Handbook is motivated by our view that easier access and an increased use of administrative data sets by researchers could dramatically improve the quantity and quality of available evidence on social programs and policies.

\newpage
\section{more}
\chapter[City of Cape Town, South Africa: Aligning Internal \\Data Capabilities with External Research Partnerships]{City of Cape Town, South Africa: Aligning Internal Data \\Capabilities with External \\Research Partnerships}
\chapterauthor{Hugh Cole, B. Kelsey Jack, Brendan Maughan-Brown, Derek Strong}
\hrulefill
\section{Summary}

The day-to-day administration of policies and programs in the City of Cape Town (CCT), South Africa generates a large amount of data. In recent years, decision-makers in CCT have begun to think strategically about how to leverage these data resources to tackle multiple and interrelated municipal policy challenges, including the sustainability of utility services (e.g., energy and water); rapid urban transformation; investments in transportation, housing, and infrastructure; the informal economy; and public safety. Specifically, CCT has made initial investments in enhancing data capabilities by adopting and implementing a data strategy and establishing a Data Science unit to facilitate greater data sharing (including through data engineering), enhanced tools for analysis (including open-source tools and data science environments), and advanced analytics. This builds on many years of investment in research, data, statistical analysis, and corporate (as opposed to fragmented) GIS capability. It also builds on the development of sophisticated policy and strategy capacity and a single policy development process for the whole organization. These steps have laid the groundwork for a broad-based effort to adopt evidence-based policymaking for greater impact and more cost-effective solutions.

Evidence-based policymaking is built on research informed by policy-relevant data sets. Practically, this requires making data available to researchers both within the municipal government and outside of it, primarily but not exclusively in academia. Data access presents numerous new challenges associated with the scale and scope of administrative data and the human resource and technological capabilities to use and share data. Specific concerns, common to many administrative contexts but particularly challenging at the municipal level, include (i) the security challenges of data sharing, (ii) the legal risks associated with greater data access, and (iii) the time and resource investment required to maintain the data architecture and governance systems that enable the sharing and use of data by various actors.

CCT maintains active research collaborations with numerous partners, which have helped to identify policy and program strengths and areas for improvement. This chapter emerges out of such a collaboration, between CCT and researchers at the Abdul Latif Jameel Poverty Action Lab (J-PAL) and University of California, Santa Barbara (UCSB), and reflects contributions from both sides of the research-policy interface. The authors describe ongoing efforts to develop a more streamlined system for cataloging, accessing, and sharing administrative data with external researchers and with analysts and decision-makers within CCT. The partnership has worked together over the past two years to advance CCT’s vision for streamlined data sharing. Bringing researchers into the planning and implementation process helps to ensure that data sharing solutions work for both policy and research. The end goal is a single, cloud-based, data sharing platform for both public-use and restricted-use data sets, documented in a browsable catalog with standardized metadata. The single platform would be used by researchers both inside and outside of the city government. A streamlined process for research permission and data access would increase researcher accountability, including the reciprocal sharing of research output, analysis code, and cleaned and new data sets. The work is still in progress and the descriptions in this chapter reflect the evolving situation.

\section{Introduction}
\subsection{Motivation and Background}

Municipalities in South Africa play numerous roles: first, providing democratic and accountable government for local communities; second, ensuring service provision to communities in a sustainable manner; third, promoting social and economic development; fourth, encouraging the involvement of communities and community organizations in the matters of local government. The City of Cape Town is no exception. It serves a population of over 4 million people and provides a mix of basic services (electricity, water, sanitation, and refuse removal) and supporting services (transport, housing, safety, emergency services, primary healthcare, environmental health, community development, environmental services, and digital infrastructure) with a 2020 operating budget of around US\$3 billion. Relative to many municipalities around the world, CCT’s data systems are well organized and well maintained. For example, all formal commercial and residential properties are registered on cadastral maps and assigned a unique parcel number to which other municipal records, including property taxes, water, electricity and refuse billing, and other services are linked. This has facilitated administrative innovations, such as consolidated billing: many businesses and residents receive a single bill for their municipal account, which streamlines the process of collections and accounting. However, these data remain under-utilized for purposes other than administration and operations. Specifically, while capacity for data analytics and research within CCT continues to grow, even internal staff struggle to identify, obtain, and process the necessary data.

\subsubsection{Test subsubsection}

More text here in this subsubsection. 

%\include{Introduction.tex}
%\include{
\end{document}
